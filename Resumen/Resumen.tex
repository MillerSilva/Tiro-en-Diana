\documentclass[10pt,a4paper]{article}
%\usepackage[english,spanish]{babel}
\usepackage{indentfirst}
\usepackage{anysize} % Soporte para el comando \marginsize
%\marginsize{1.5cm}{1.5cm}{0.5cm}{1cm}
\marginsize{2,5cm}{1,8cm}{4cm}{1,7cm}
\usepackage[psamsfonts]{amssymb}
\usepackage{amssymb}
\usepackage{amsfonts}
\usepackage{amsmath}
\usepackage{amsthm}
\usepackage{stackrel}
\usepackage{graphicx}
\usepackage[colorinlistoftodos]{todonotes}
%Color a las referencias
\usepackage[colorlinks=true, allcolors=blue]{hyperref}
\usepackage[spanish]{babel}
\selectlanguage{spanish}
\usepackage[utf8]{inputenc} 

\usepackage{multicol}
\renewcommand{\thepage}{}
\columnsep=7mm

%%%%%%%%%%%%%%%%%%%%%%%%%%%%%%%%%%%%%%%%
\newtheorem{definicion}{Definici\'on}[section]
\newtheorem{teorema}{Teorema}[section]
\newtheorem{prueba}{Prueba}[section]
\newtheorem{prueba*}{Prueba}[section]
\newtheorem{corolario}{Corolario}[section]
\newtheorem{observacion}{Observaci\'on}[section]
\newtheorem{lema}{Lema}[section]
\newtheorem{ejemplo}{Ejemplo}[section]
\newtheorem{solucion*}{Soluci\'on}[section]
\newtheorem{algoritmo}{Algoritmo}[section]
\newtheorem{proposicion}{Proposici\'on}[section]

\linespread{1.4} \sloppy

\newcommand{\R}{\mathbf{R}}
\newcommand{\N}{\mathbf{N}}
\newcommand{\C}{\mathbb{C}}
\newcommand{\Lr}{\mathcal{L}}
\newcommand{\fc}{\displaystyle\frac}
\newcommand{\ds}{\displaystyle}

\DeclareMathOperator{\Dom}{Dom}

%%%%%%%%%%%%%%%%%%%%%%%%%%%%%%%%%%%%%%%%

\renewcommand{\thefootnote}{\fnsymbol{footnote}}
\usepackage{url}
\usepackage{hyperref}

\begin{document}
\begin{center}
 {\Large \textbf{TIRO EN DIANA}}
\end{center}
\begin{center}
 Gustavo Lozano$^{1}$, Miller Silva$^{2}$, Ayrton Coronado$^{3}$, Mirian Geronimo$^{4}$, Guillermo Borjas$^{5}$ \vskip5pt
 {\it Facultad de Ciencias$^1$, Universidad Nacional de Ingenier\'{\i}a$^1$\\}\vskip5pt
 Email: glozanoa@uni.pe$^{1}$, miller.silva.m@uni.pe$^{2}$, acoronadoh@uni.pe$^{3}$, mgeronimoa@uni.pe$^{4}$, gborjasc@uni.pe$^{5}$
\end{center}
%\maketitle 
\vspace*{1cm}
\begin{abstract}

\noindent En la Academia General de Zaragoza se programa una actividad de tiro para los Caballeros Cadetes.
Para cierto Caballero Cadete, se dispone de 12 dianas  y en cada diana realizará 12 disparos. Las puntuaciones de cada disparo se ordenan en una matriz $A \in \mathbb{R}^{12\times 12}$ (\textit{matriz de puntuación de la semana 1}). Suponiendo que \textit{la matriz de puntuación de la semana $n$ }esta dada por $A^n$ y la \textit{puntuación final de tiro de la semana $n$ }está dada por la norma de la matriz ($||A^n||_1 \mbox{ o } ||A^n||_\infty$), se desea hallar la puntuación final de tiro de la semana 10 usando ambas normas y concluir con qué norma sale más beneficiado el Caballero Cadete.

La dificultad de este trabajo se centra en encontrar una expresión general para la potencia $n-$ésima de la matriz $A$, para esto vamos a usar la matriz de Jordan de $A$ ($J_A$) ya que se cumple $A^n=P^{-1}J^n_AP$, donde la potencia  $J^n_A$ es más fácil de hallar comparado a $A^n$. Para hallar la matriz de Jordan de $A$, usaremos la transformación de Householder.

\end{abstract}

\begin{quotation}
	{\small
		\noindent\textbf{Palabras Clave:} \\ 
	Matriz de puntuación, Puntuación final , Matriz de Jordan, Transformación de Householder\\
	}
\end{quotation}

\renewcommand{\abstractname}{Abstract}
\begin{abstract}
	\noindent In the Academia General de Zaragoza a shooting activity is scheduled
	for the Cadet Knights. For a certain Cadet Knight, they have
	12 targets and in each target will make 12 shots. The scores of each
	shot are ordered in a matrix $A \in \mathbb{R}^{12\times 12}$  (\textit{scoring matrix of the
		Week 1 }). Assuming that the \textit{score matrix of the week $n$ }is
	given by $A^n$ and the \textit{final punctuation of the week n } is given by the
	standard of the matrix ($||A^n||_1 \mbox{ or } ||A^n||_\infty$), we want to find the punctuation
	end of shot of week 10 using both standards and conclude what standard
	The Cadet Knight benefits more.
	The difficulty of this work is focused on finding a general expression
	for the $n-th$ power of matrix $A$, for this we are going to use the matrix
	of Jordan from $A (J_A)$ since $A^n=P^{-1}J^n_AP$ is met, where the $J^n_A$ power is easier to find compared to $A$. To find the Jordan matrix of $A$,
	we will use the Householder transformation.
	
\end{abstract}


\begin{quotation}
	{\small
		\noindent \textbf{Keywords:} \\ 
		Scoring matrix, Final punctuation , Jordan's matrix, Householder transformation\\
	}
\end{quotation}
\end{document}