\documentclass[10pt,a4paper]{article}
\usepackage[utf8]{inputenc}
\usepackage[spanish]{babel}
\usepackage{amsmath}
\usepackage{amsfonts}
\usepackage{amssymb}
\usepackage{makeidx}
\usepackage{graphicx}
\usepackage[left=2cm,right=2cm,top=2cm,bottom=2cm]{geometry}
%%%%%%%%% Theorems
\newtheorem{mytheo}{Teorema}[section]
\newtheorem{mynote}{Nota}[section]
\newcommand{\ds}{\displaystyle}
\title{Resumen Teórico}
\begin{document}

\section{Uso de la Forma Canónica de Jordan en cálculos de funciones de matriciales}

Dado que en muchas ocaciones es necesario calcular $f(A)$, donde $f:\mathbb{R}^{n\times n}\rightarrow V$ una función y $A\in\mathbb{R}^{n\times n}$, esto nos motiva a tratar de calcular $f(A)$ de manera eficiente(e inteligente), es decir en vez de hacer calculos con $A$, lo que haremos es buscar un matriz equivalente a $A$, pero tal que los calculos sean muy sencillos(o al menos más sencillos que realizarlos con la matriz original). Asi que esto nos motiva a buscar dicha matriz, por ejemplo:

Si $A$ fuera diagonalizable, entonces existe $P\in\mathbb{R}^{n\times n}$ inversible tal que:
$$A = PDP^{-1}$$,
donde $D$ es una diagonal. Asi $A^{k} = PD^{k}P^{-1}$, por lo tanto si requerimos $A^{k}$, el calculo sera hecho de manera indirecta, es decir calcular $D^{k}$(que es muchisimo más simple) y luego calcular $PD^{k}P^{-1}$.

Pero es claro que no todas las matrices son diagonalizables, por lo que es necesario buscar alguna matriz sencilla equivalente con $A$ tal que nos permita respinder sobre cuestiones que involucren a $A$.

Dicha matriz es la forma canónica de Jordan y está dada  por:

$$J(A) = \begin{bmatrix}
J(A,\lambda_{1})	&	O	&	O	&	\ldots	&	O\\
O	&	J(A,\lambda_{2})	&	O	&	\ldots	&	O\\
\vdots	&	\vdots	&	\vdots	&	\ddots	&	\vdots\\
O	&	O	&	O	&	\ldots	&	J(A, \lambda_{k})
\end{bmatrix}$$ 

Donde $\{\lambda_{1},\lambda_{2},\ldots ,\lambda_{k}\}\in \wedge (A)$ y $J(A,\lambda_{k})$ es el bloque de Jordan esociado a $\lambda_{i}$

Esta matriz está relacionada con $A$, mediante una matriz $Q\in\mathbb{R}^{n\times n}$ inversible tal que:
$$A = QJ(A)Q^{-1}$$

Ahora si nos proponemos a calcular $e^{A}$, pero sabemos:
$$e^{x} = \sum_{k=0}^{\infty}\frac{x^{k}}{k!}$$
Entonces definimos $e^{A} = \sum_{k=0}^{\infty}\frac{A^{k}}{k!}$, si es que $\sum_{k=0}^{\infty}\frac{A^{k}}{k!}$ converge, asi:

\begin{align*}
e^{A} 	&= \sum_{k=0}^{\infty}\frac{A^{k}}{k!} \\
		&= \sum_{k=0}^{\infty}\frac{QJ(A)^{k}Q^{-1}}{k!}\\
		&= Q\left(\sum_{k=0}^{\infty}\frac{J(A)^{k}}{k!}\right)Q^{-1}\\
		& = Q e^{J(A)}Q^{-1}
\end{align*}

Pero $e^{J(A)}$ es más sencillo que calcular $e^{A}$, por lo cual la forma canónica de $A$, nos facilita de algún  modo el trabajo.
Del mismo modo podemos calcular $\cos (A)$ o $\sin (A)$:
\begin{align*}
	Como &\\
		& sin x = \sum_{k=0}^{\infty} \frac{x^{4k+1}}{(4k+1)!} - \frac{x^{4k+3}}{(4k+3)!}\\
		& cos x = \sum_{k=0}^{\infty} \frac{x^{4k}}{(4k)!} - \frac{x^{4k+2}}{(4k+2)!}
\end{align*}

Así:
\begin{align*}
	\sin (A) & = \sum_{k=0}^{\infty} \frac{A^{4k+1}}{(4k+1)!} - \frac{A^{4k+3}}{(4k+3)!}\\
			&= \sum_{k=0}^{\infty} \frac{QJ(A)^{4k+1}Q^{-1}}{(4k+1)!} - \frac{QJ(A)^{4k+3}Q^{-1}}{(4k+3)!}\\
			&= Q\left(\frac{J(A)^{4k+1}}{(4k+1)!} - \frac{J(A)^{4k+3}}{(4k+3)!}\right)Q^{-1}\\\\
	\cos (A) &= Q\left(\frac{J(A)^{4k}}{(4k)!} - \frac{J(A)^{4k+2}}{(4k+2)!}\right)Q^{-1}
\end{align*}

Gneneralizando, dada $f\in\mathbb{C}^{\infty}$, entonces;
\begin{align*}
	f(x) & = \sum_{k=0}^{\infty}\frac{f^{(k)}(0)}{k!}x^{k}\\
	\Rightarrow f(A) &= \sum_{k=0}^{\infty}\frac{f^{(k)}(0)}{k!}A^{k}\\
		&= Q\left(\sum_{k=0}^{\infty}\frac{f^{(k)}(0)}{k!}J(A)^{k}\right)Q^{-1}
\end{align*}

\section{Forma Canónica de Jordan}

\begin{mytheo}[Descomposición]
Sea $T:V\rightarrow V, \dim V = n$. Entonces existen $U, W\subset V$ invariantes por $T$ tal que:
\begin{enumerate}
	\item $V = U\oplus W$
	\item $T_{|U}: U\rightarrow U$ nilpotente y $T_{|W}: W\rightarrow W$ inversible.
\end{enumerate}
\end{mytheo}

\textbf{Ver prueba en Jordan-Teoria}

\begin{mynote}
Como $T_{|U}: U\rightarrow U$ es nilpotente, entonces existe $q\in\mathbb{N}/\quad \left(T_{|U}\right)^{q} = 0\quad\wedge\quad \left(T_{|U}\right)^{q-1} \neq 0$. Ahora hallamos un $v\neq 0/ \left(T_{|U}\right)^{q-1}v \neq 0$. ASi definimos:
$$\beta = \left\{ (T_{|U})^{q-1}v, (T_{|U})^{q-2}v, \ldots, (T_{|U})v, v\right\}$$
Las cual es una base de $U$. Como $W$ es invariante por $T$ podemos definir $T_{|W}:W\rightarrow W$. Y volviendo a aplicar el teorema anterior $\exists u_{2}, w_{2}\subset W$ invariantes por $T$, tal que:

$W = U_{2}\oplus W_{2}$, tal que $T_{|U_{2}}:U_{2}\rightarrow U_{2}$ nilpotente y $T_{|W_{2}}:W_{2}\rightarrow W_{2}$ inversible.

Como $T_{|U_{2}}:U_{2\rightarrow U_{2}}$ nilpotente, $\exists q_{2}\in\mathbb{N}/\quad (T_{|U_{2}})^{q2} = 0\quad\wedge\quad (T_{|U_{2}})^{q2-1} \neq 0$, además $q_{1} = q \geq q_{2}$ (\textbf{VER DEMOSTRACION EN Jordan-Teoria})

Ahora hallando un $v_{2}\neq 0/\quad (T_{|U_{2}})^{q_{2}-1}v_{2}\neq 0$

De esto definimos:
$$\beta_{2} = \left\{ (T_{|U_{2}})^{q_{2}-1}v_{2}, (T_{|U_{2}})^{q_{2}-2}v_{2}, \ldots, (T_{|U_{2}})v_{2}, v_{2}\right\}$$
La cual es una base de $U_{2}$, luego de manera recursiva podemos tener:
$$V = U_{1}\oplus U_{2}\oplus \oplus\ldots U_{k}$$
, donde $U_{1} = U$. Luego $\beta = \bigcup_{i=1}^{k}\beta_{i}$ es base de $V$ y además:

$$J(A) = \begin{bmatrix}
[T_{|U_{1}}]_{\beta_{1}}	&	O	&	O	&	\ldots	&	O\\
O	&	[T_{|U_{2}}]_{\beta_{2}}	&	O	&	\ldots	&	O\\
\vdots	&	\vdots	&	\vdots	&	\ddots	&	\vdots\\
O	&	O	&	O	&	\ldots	&	[T_{|U_{k}}]_{\beta_{k}}
\end{bmatrix}$$


Donde:

$$
[T_{|U_{j}}]_{\beta_{j}} = \begin{bmatrix}
0	&	1	&	0	&	0	&	0	&	\ldots	&	0\\
0	&	0	&	1	&	0	&	0	&	\ldots	&	0\\
0	&	0	&	0	&	1	&	0	&	\ldots	&	0\\
0	&	0	&	0	&	0	&	1	&	\ldots	&	0\\
\vdots	&	\vdots 	&	\vdots	&	\vdots	&	\vdots &	\ddots	&	\vdots\\
0	&	0	&	0	&	0	&	1	&	\ldots	&	1
\end{bmatrix}$$
\end{mynote}


\begin{mytheo}[Forma Canónica de Jordan]

\end{mytheo}


\section{Eigenvalue-Eigenvector computation}
Dada una matriz $A\in \mathbb{R}^{n\times n}$, este método calcula el mayor valor propio de $A$ y el vector propio asociado a este vector propio.

\subsection{Convergencia del Método Potencia}
\begin{itemize}
	\item Si cada $\lambda\in\wedge (A)$ tieene mutiplicidad $1$, luego podemos ordenarlos(los valores propios), de la siguiente forma:
	$$\vert \lambda_{1}\vert >\vert\lambda_{2}\vert >\ldots >\vert\lambda_{n}\vert$$
	
	Ahora sea $\{v_{1}, v_{2},\ldots, v_{n}\}$ los vectores propios asociados a los valores propios(respectivamente, $Av_{i} = \lambda_{i}v_{i}$). Así $\{v_{1}, v_{2},\ldots, v_{n}\}$ es una base de $\mathbb{R}^{n\times 1}$.Por lo tanto para $q^{(0)}\in\mathbb{R}^{n\times 1}$. Por lo tanto para $q^{(0)}\in\mathbb{R}^{n\times 1}$, tenemos:
	$$q^{(0)} = \sum_{k=1}^{n}c_{k}v_{k}$$
	
Asi podemos definir la siguiente suceción (la cual probaremos que converge a x1)
$$q^{(k+1)} = \frac{1}{\phi_{k}}Aq^{(k)},\quad donde\;\phi_{k} = \Vert Aq^{(k)}\Vert$$

\begin{align*}
	q^{(k+1)} &= \frac{1}{\phi_{k}}A\left(\frac{1}{\phi_{k-1}}Aq^{(k-1)}\right)\\
		&= \frac{1}{\phi_{k}\phi_{k-1}}A^{2}q^{(k-1)}\\
		&= \frac{1}{\phi_{k}\phi_{k-1}\phi_{k-2}}A^{3}q^{(k-2)}\\
		&\vdots\\
		&= \frac{1}{\phi_{k}\phi_{k-1}\ldots\phi_{0}}A^{k+1}q{(0)}\\\\
	q^{(k+1)} &= \left(\prod_{i=0}^{k}\phi_{i}^{-1}\right)A^{k+1}q^{(0)}\\
		&= \phi A^{k+1}\left(\ds\sum_{j=1}{n}c_{j}x_{j}\right),\quad\phi = \left(\prod_{i=0}^{k}\phi_{i}^{-1}\right)\\
		&= \phi \ds\sum_{j=1}{n}c_{j}A^{k+1}x_{j}\\
		&= \phi \ds\sum_{j=1}{n}c_{j}\lambda_{j}^{k+1}x_{j}\\
		&= \phi\lambda_{1}^{k+1}\left(c_{1}x_{1} + \ds\sum_{j=2}^{n}c_{j}\left(\frac{\lambda_{j}}{\lambda_{1}}\right)^{k+1}x_{j}\right)
\end{align*}


Como $\left\vert\frac{\lambda_{j}}{\lambda_{1}}\right\vert < 1,\quad j=2-n$, entonces:
$$\lim_{k\rightarrow\infty}q^{(k+1)} = \left(\lim_{k\rightarrow\infty}\prod_{i=0}^{k}\lambda_{j}^{k+1}\right)(c_{1}x_{1}+0)$$
$$\therefore\quad\lim_{k\rightarrow\infty}q{(k+1)} = c_{1}ax_{1},\quad donde\quad a = \left(\lim_{k\rightarrow\infty}\prod_{i=0}^{k}\lambda_{j}^{k+1}\right)$$


Ahora vamos a construir una sucesión que converja al valor porpio $\lambda_{1}$.
\begin{equation}\label{eigenvalue_iter}
u^{(k+1)} = \frac{1}{\phi_{k}}\frac{(Aq^{(k)})_{j}}{q^{(k)}_{j}}
\end{equation}
donde $(Aq^{(k)})_{j}$ es el j-esimo elemento de $Aq^{(k)}$ y $q^{(k)}_{j}$ es el j-esimo elemento de $q^{(k)}$

Vamos a denotar $x_{ij}$ al j-esimo elemento de $x_{i}$. Como 
\begin{align*}
	Aq^{(k)} &= \phi_{k}q^{(k+1)}\\
	\Rightarrow	Aq^{(k)}_{j} &= \phi_{k}q^{(k+1)}_{j}\\
	\Rightarrow	u^{(k+1)} &= \frac{1}{\phi_{k}}\frac{\phi_{k}(q^{(k+1)})_{j}}{q^{(k)}_{j}} = \frac{q_{j}^{(k+1)}}{q_{j}^{(k)}}\\
\end{align*}

Como 
$$ q{(m)} = \phi\lambda_{1}^{m+1}\left(c_{1}x_{1} + \sum_{i=2}^{n}c_{i}\left(\frac{\lambda_{j}}{\lambda_{1}}\right)^{m+1}x_{i}\right),\quad m\in\mathbb{N}$$

En $(\ref{eigenvalue_iter})$
\begin{align*}
	u^{(k+1)} &= \frac{\phi\lambda_{1}^{k+1}\left(c_{1}x_{1j} + \ds\sum_{i=2}^{n}c_{i}\left(\frac{\lambda_{j}}{\lambda_{1}}\right)^{k+1}x_{ij}\right)}{\phi\lambda_{1}^{k}\left(c_{1}x_{1j} + \ds\sum_{i=2}^{n}c_{i}\left(\frac{\lambda_{j}}{\lambda_{1}}\right)^{k}x_{ij}\right)}\\
		&= \lambda_{1}\left[\frac{c_{1}x_{1j} + \ds\sum_{i=2}^{n}c_{i}\left(\frac{\lambda_{j}}{\lambda_{1}}\right)^{k+1}x_{ij}}{c_{1}x_{1j} + \ds\sum_{i=2}^{n}c_{i}\left(\frac{\lambda_{j}}{\lambda_{1}}\right)^{k}x_{ij}}\right]\\
	\Rightarrow	\lim_{k\rightarrow\infty}u^{k+1} &= \lambda_{1}\lim_{k\rightarrow\infty}	\frac{c_{1}x_{1j} + \ds\sum_{i=2}^{n}c_{i}\left(\frac{\lambda_{j}}{\lambda_{1}}\right)^{k+1}x_{ij}}{c_{1}x_{1j} + \ds\sum_{i=2}^{n}c_{i}\left(\frac{\lambda_{j}}{\lambda_{1}}\right)^{k}x_{ij}} = \lambda_{1}\left(\frac{c_{1}x_{1}}{c_{1}x_{1}}\right) = \lambda_{1}
\end{align*}

	\item Caso general\\
	
	Sea $\wedge (A) = \{\lambda_{1},\lambda_{2},\ldots ,\lambda_{k}\}$ y $m_{1},m_{2}, \ldots , m_{k}\in\mathbb{N}$ mutiplicidades geométricas de $\lambda_{1},\lambda_{2},\ldots ,\lambda_{k}$ respectivamente(No necesariamente $V = V_{1}\oplus V_{2}\oplus V_{3}\oplus\ldots\oplus V_{k}, v_{i}$ espacio propio de $\lambda_{i}$).
	
	Sin perdida de la generalidad, sea:
	$$\vert\lambda_{1}\vert > \vert\lambda_{2}\vert > \vert\lambda_{3}\vert > \ldots\vert\lambda_{k}\vert$$
	
	Ahora sea $q^{(0)}\in\left<\bigcup_{i=1}^{k}V_{i}\right>$
	$$\Rightarrow\quad q^{(0)} = \sum_{i=1}^{m_{1}}c_{i}^{(1)}v_{i}^{(1)} + \sum_{r=1}^{k}\sum_{j=1}^{m_{r}}c_{j}^{(r)}v_{j}^{(r)}$$
	donde $v_{j}^{(i)}$: vector j-esimo del espacio propio $v_{i}$


\end{itemize}
\end{document}
