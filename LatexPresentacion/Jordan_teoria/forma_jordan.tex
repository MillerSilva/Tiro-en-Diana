\documentclass[10pt,a4paper]{article}
\usepackage[utf8]{inputenc}
\usepackage[spanish]{babel}
\usepackage{amsmath}
\usepackage{amsfonts}
\usepackage{amssymb}
\usepackage{graphicx}
\usepackage[left=2cm,right=2cm,top=2cm,bottom=2cm]{geometry}
\newtheorem{mydef}{Definicion}[section]
\newtheorem{mynote}{Nota}
\newtheorem{mytheo}{Teorema}
\newtheorem{myexamp}{Ejemplo}
\newtheorem{mycorol}{Corolario}
\title{Forma Canónica de Jordan}
\begin{document}
\maketitle


\section{Forma Canónica de Jordan para Operadores Nilpotentes}
\begin{mydef}[Operadores Lineales Nilpotentes]
	El operador $T:V\rightarrow V$ es nilpotente si $T^{p} = 0$, para algún $p\in\mathbb{N}$. Además se dice que $k\in\mathbb{N}$, es el índice de nilpotencia de $T$ si:
$$T^{k-1}\neq 0 \wedge T^{k} = 0$$

\end{mydef}

\begin{mytheo}\label{theo_nilpotentes}
	Sea $T:V\rightarrow V$, $V$ un $\mathbb{C}$ espacio vectorial, $\dim V = n$, $T$ nilpotente de índice $q$. Luego para $v\in V/ T^{q-1}v\neq 0$, tenemos:
	\begin{enumerate}
		\item El conjunto $\{T^{q-1}v, T^{q-2}v,\ldots,Tv, v\}$ es LI.
		\item S = $\left<\{T^{q-1}v, T^{q-2}v,\ldots,Tv, v\}\right>$ es invariante por $T$.
		\item Existe un subespacio $U$ de $V$, invariante por $T/$
		$$V = S\oplus U$$
	\end{enumerate}
\end{mytheo}

\textbf{Prueba}
\begin{enumerate}
	\item \begin{equation}\label{li_equation}
		a_{0}v + a_{1}Tv + a_{2}T^{v} + \ldots + a_{q-1}T^{q-1}v = 0
	\end{equation}
	\begin{align*}
		\Rightarrow T^{q-1}\left(a_{0}v + a_{1}Tv + a_{2}T^{v} + \ldots + a_{q-1}T^{q-1}v\right) & = 0\\
		a_{0}T^{q-1}v &= 0\\
		a_{0} &= 0
	\end{align*}
	De $(\ref{li_equation})$
	$$a_{1}Tv + a_{2}T^{v} + \ldots + a_{q-1}T^{q-1}v = 0$$
	\begin{align*}
		\Rightarrow T^{q-2}\left(a_{1}Tv + a_{2}T^{v} + \ldots + a_{q-1}T^{q-1}v\right) & = 0\\
		a_{1}T^{q-1}v &= 0\\
		a_{1} &= 0
	\end{align*}
	De forma similar tenemos:
		$$a_{0} = a_{1} = a_{2} = \ldots = a_{q-1} = 0$$
		$$\therefore \{T^{q-1}v, T^{q-2}v,\ldots,Tv, v\},\quad son \:LI$$
	\item Sea $ u\in S$
		\begin{align*}
			\Rightarrow u &= \sum_{k=0}^{q-1}a_{k}T^{k}v\\
			\Rightarrow Tu &=  \sum_{k=0}^{q-1}a_{k}T^{k+1}v = \sum_{k=0}^{q-2}a_{k}T^{k+1}v\in S
		\end{align*}
		$$\therefore T(S)\subset S$$
	\item VER
\end{enumerate}

\begin{mycorol}
	Del teorema $(\ref{theo_nilpotentes})$
	\begin{align*}
		S &= \left<\{T^{q-1}v, T^{q-2}v,\ldots,Tv, v\}\right>\\
		&\Rightarrow\dim (S) = q\\			
	\end{align*}

	Luego 
	\begin{align*}
		T(S) &= \left<\{T^{q-1}v, T^{q-2}v,\ldots,T^{2}v, Tv\}\right>\\
		&\Rightarrow\dim T(S) = q-1\\\\
		T^{2}(S) &= \left<\{T^{q-1}v, T^{q-2}v,\ldots,T^{3}v, T^{2}v\}\right>\\
		&\Rightarrow\dim T^{2}(S) = q-2
	\end{align*}
	Inductivamente 
	\begin{align*}
		T^{r}(S) &= \left<\{T^{q-1}v, T^{q-2}v,\ldots,T^{r+1}v, T^{r}v\}\right>\\
		&\Rightarrow\dim T^{r}(S) = q-r\\			
	\end{align*}
\end{mycorol}

\begin{mynote}
	Como $S = S_{1}$ es invariante por $T$, podemos definir el operador $T$ sobre $S_{1}$, es decir, $T:S_{1}\rightarrow S_{1}$ cuyo indice de nilpotencia es $q_{1} = q$.
Del teorema $(\ref{theo_nilpotentes})$, $\beta_{1} = \{T^{q-1}v, T^{q-2}v,\ldots,Tv, v\}$ es base de $S_{1}$.
Así 
\begin{align*}
	[T]_{\beta_{1}} &= 
	\left[
	\left[T\left(T^{q_{1}-1}v\right)\right], 
	\left[T\left(T^{q_{1}-2}v\right)\right],
	\left[T\left(T^{q_{1}-3}v\right)\right],
	\ldots,
	\left[T\left(Tv\right)\right],
	\left[T\left(v\right)\right]	
	\right]\\\\
	[T]_{\beta_{1}} &= 
	\begin{bmatrix}
		0	&	1	&	0	&	0	&	\ldots	&	0\\
		0	&	0	&	1	&	0	&	\ldots	&	0\\
		0	&	0	&	0	&	1	&	\ldots	&	0\\
		\vdots	&	\vdots	&	\vdots	&	\vdots	&	\ddots	&	\vdots\\
		0	&	0	&	0	&	0	&	\ldots	&	1\\
		0	&	0	&	0	&	0	&	\ldots	&	0\\
	\end{bmatrix} = [T_{|S1}]_{\beta_{1}}
\end{align*}
Además como $U = U_{1}$ invariante por $T$, podemos definir $T$ sobre $U_{1}$, $T: U_{1}\rightarrow U_{1}$, y volver a aplicar el \textit{teorema}$(\ref{theo_nilpotentes})$, es decir existe $S_{2}$ y $U_{2}$ subespacios de $U_{1}$, invariantes por $T$/
$$U_{1} = S_{2}\oplus U_{2}$$
Repitiendo lo anterior pero con $S = S_{2},\exists u\in V/T^{q_{2}-1}u\neq 0$, donde $q_{2}$ es el indice de nilpotencia de $T_{|S} = T:S\rightarrow S$, asi:
\begin{align*}
	\beta_{2} &= \left\{T^{q_{2}-1}u, T^{q_{2}-2}u,\ldots,Tu, u\right\}
\end{align*}
$$[T]_{\beta_{2}} = 
\begin{bmatrix}
	0	&	1	&	0	&	0	&	\ldots	&	0\\
	0	&	0	&	1	&	0	&	\ldots	&	0\\
	0	&	0	&	0	&	1	&	\ldots	&	0\\
	\vdots	&	\vdots	&	\vdots	&	\vdots	&	\ddots	&	\vdots\\
	0	&	0	&	0	&	0	&	\ldots	&	1\\
	0	&	0	&	0	&	0	&	\ldots	&	0\\
\end{bmatrix} = [T_{|S2}]_{\beta_{2}}$$
$$$$
Podemos continuar descomponiendo los $U_{k}$, por el teorema anterior, hasta obtener:
$$V = S_{1}\oplus S_{2}\oplus \ldots \oplus S_{k}$$
De esto $\beta = \bigcup_{i=1}^{k}\beta_{i}$ es base de $V$, donde $\beta_{i}$ es base de $S_{i}$.
Además
$$\begin{bmatrix}
	\left[T_{|S_{1}}\right]_{\beta_{1}}	&	O	&	\ldots	&	O\\
	O	&	\left[T_{|S_{2}}\right]_{\beta_{2}}	&	\ldots	&	O\\
	\vdots	&	\vdots	& \ddots	& \vdots\\
	O	&	O	&	\ldots	&	\left[T_{|S_{k}}\right]_{\beta_{k}}
\end{bmatrix}$$

Note que el bloque $\left[T_{|S_{k}}\right]_{\beta_{k}}$ es más grande (o igual) que el bloque $\left[T_{|S_{m}}\right]_{\beta_{m}}$, para $k > m$, pues $q_{k}\geq q_{m}$.

\end{mynote}


\begin{mytheo}[Estructura de la matriz de un Operador Lineal]
Sea $T:V\rightarrow V, \:dim V = n$, un operador lineal nilpotente de índice $q$, entonces:
\begin{enumerate}
	\item $\exists q_{1} = q, q_{2}, q_{3},\ldots, q_{r}\in\mathbb{N}\diagup\quad q_{1}\geq q_{2}\geq q_{3}\geq\ldots\geq q_{r}$
	\item HERE SEE, WHAT HAPPEN WITH ALIGN ENVIROMENT
	%\begin{align*}
	%& \exists v_{1}, v_{2},\ldots, v_{r}\in V\diagup \\
	%\beta & = \left\{ \\
	%& T^{q_{1}-1} v_{1}, T^{q_{1}-2} v_{1}, \ldots, T v_{1}, v_{1},\\
	%& T^{q_{2}-1} v_{2}, T^{q_{2}-2} v_{2}, \ldots, T v_{2}, v_{2},\\
	%& T^{q_{3}-1} v_{3}, T^{q_{3}-2} v_{3}, \ldots, T v_{3}, v_{3},\\
	%& \vdots \\
	%& T^{q_{r}-1} v_{r}, T^{q_{r}-2} v_{r}, \ldots, T v_{r}, v_{r}\right\}
	%\end{align*}
	\item
	$$T^{q_{1}} v_{1} = T^{q_{2}}v_{2} = \ldots = T^{q_{r}}v_{r} = 0$$
\end{enumerate}
\end{mytheo}

\begin{myexamp}
	Hallar la forma canónica de Jordan de la matriz nilpotente.
	$$\begin{pmatrix}
		0	&	1	&	0	&	-1\\
		1	&	0	&	-1	&	0\\
		0	&	1	&	0	&	-1\\
		-1	&	0	&	1	&	0
	\end{pmatrix}$$
\end{myexamp}
\textbf{Solución}\\
Sea $V = \mathbb{R}^{4\times 1}$, definiendo $T:V\rightarrow V$ como:
\begin{align*}
	T: V &\rightarrow V\\
	X &\mapsto TX = AX
\end{align*}
Luego $[T]_{\beta} = A, \quad\beta$ base canónica de $\mathbb{R}^{4\times 1}$
Hallando el indice de nilpotencia de $T$(que es igual a hallar el indice de nilpotencia de $A$)
$$A^{2} = 
\begin{bmatrix}
2	&	0	&	-2	&	0\\
0	&	0	&	0	&	0\\
2	&	0	&	-2	&	0\\
0	&	0	&	0	&	0
\end{bmatrix},\qquad
A^{3} = 0
$$
Por lo tanto el indice de nilpotencia es $q_{1} = q = 3$.
Hallando $v_{1}\in\mathbb{R}^{4\times 1}\diagup T^{q_{1}-1}v_{1}\neq 0$

Tomando $v_{1} = 
\begin{bmatrix}
	0	\\
	1	\\
	1	\\
	1
\end{bmatrix}
$, tenemos $T^{q_{1}-1}v_{1} = T^{2}v_{1} = A^{2}v_{1} = 
\begin{bmatrix}
	-2	\\
	0	\\
	-2	\\
	0
\end{bmatrix}\neq 0
$

Asi \begin{align*}
	S_{1} 	&= \left<\{T^{2}v_{1}, Tv_{1}, v_{1}\}\right>\\
			&=\left<\left\{
			\begin{bmatrix}
			-2	\\
			0	\\
			-2	\\
			0
			\end{bmatrix}, 
			\begin{bmatrix}
			0	\\
			-1	\\
			0	\\
			1
			\end{bmatrix},
			\begin{bmatrix}
			0	\\
			1	\\
			1	\\
			1
			\end{bmatrix}\right\}\right>
\end{align*}
Luego por \textit{teorema}$\left(\ref{theo_nilpotentes}\right)$, $\exists U_{1}\subset V = \mathbb{R}^{4\times 1}$ subspacio invariante por $T\diagup$
$$\mathbb{R}^{4\times 1} = V = S_{1}\oplus U_{1}$$

Por lo que $\dim U_{1} = 1$, asi $S_{2} = U_{1}$, con índice de nilpotencia $q_{2} = 1$.

Hallando $v_{2}\in V\diagup T^{q2}v_{2} = 0$\\
Resolviendo $Av = 0$, tenemos 
$v = t
\begin{bmatrix}
1	\\
0	\\
1	\\
0
\end{bmatrix}
+ s
\begin{bmatrix}
0	\\
1	\\
0	\\
1
\end{bmatrix}$

Tomando $v_{2} = 
\begin{bmatrix}
0\\
1\\
0\\
1
\end{bmatrix}$, pues 
$\begin{bmatrix}
1\\
0\\
1\\
0
\end{bmatrix}\in S_{1}$

Finalmente $\beta = \left\{T^{2}v_{1}, Tv_{1}, v_{1}, v_{2}\right\}$ es una base para $V$ y 

$$[T]_{\beta} = 
\begin{bmatrix}
	0	&	1	&	0	&	0\\
	0	&	0	&	1	&	0\\
	0	&	0	&	0	&	0\\
	0	&	0	&	0	&	0
\end{bmatrix}
$$

\section{Forma Canónica de Jordan (general)}
\begin{mytheo}
	sea $T:V\rightarrow V$ un operador lineal, $\dim V = n$. Entonces $\exists\;U, W\subset V$ subspacios invariantes por $T$/
	\begin{enumerate}
		\item $V = U\oplus W$
		\item $T_{|U}:U\rightarrow U$ es nilpotente y $T_{|W}:W\rightarrow W$ es inversible.
	\end{enumerate}
\end{mytheo}
\textbf{Demostración}\\
\begin{itemize}
	\item Casos triviales
			\begin{itemize}
				\item $T$ inversible, simplemente tomamos $W = V$ y $U=\emptyset$.
				\item $T$ nilpotente, tomamos $W = \emptyset$ y $U = V$.
			\end{itemize}
	\item Si $T$ no es inversible.
		$$\Rightarrow N(T) \neq \emptyset \Rightarrow N(T^{p}) \neq \emptyset,\quad p\in\mathbb{N}$$
		Por otro lado se cumple:
		\begin{align*}
		& N(I)\subseteq N(T)\subseteq N(T^{2})\subseteq \ldots\subseteq N(T^{q})\subseteq N(T^{q+1}) \subseteq\ldots\subseteq V\\
		& V \supseteq Im(I)\supseteq Im(T)\supseteq Im(T^{2})\supseteq\ldots\supseteq Im(T^{q})\supseteq Im(T^{q+1})\supseteq\ldots
		\end{align*}
		Sea $q\in\mathbb{N}$ el menor que cumple $N(T^{q}) = N(T^{q+1})$
		
		Se afirma que:
		\begin{equation}\label{nilp_index}
		N(T^{q}) =  N(T^{q+k}),\quad\forall k\in\mathbb{N}
		\end{equation}
		Primero demostrando
		$$N(T^{q+1}) = N(T^{q+2})$$
		\begin{description}
			\item[$(\subseteq)$] Directo
			\item[$(\supseteq)$] Sea $u\in N(T^{q+2})$
			\begin{align*}
				T^{q+2}u &= 0\\
				T^{q+1}(Tu) &= 0\\
				Tu\in N(T^{q+1}) &= N(T^{q})\\
				T^{q}(Tu) &= 0\\
				T^{q+1}u &= 0\\
				\Rightarrow u &\in N(T^{q+1})
			\end{align*}
		\end{description}
		Probando $\left(\ref{nilp_index}\right)$ por inducción matemática.
		\begin{itemize}
			\item $k = 1$ Cumple
			\item Suponiendo válido para k $\left(N\left(T^{q}\right) = N\left(T^{q+k}\right) \right)$
			
			\item Para $k+1$, debemos probar $N\left(T^{q}\right) = N\left(T^{q+k+1}\right)$
				\begin{description}
					\item[$(\subseteq)$] Directo
					\item[$(\supseteq)$] Sea $u\in N\left(T^{q+k+1}\right)$
						\begin{align*}
							T^{q+k+1}u &= 0\\
							T^{q+k}(Tu) &= 0\\
							Tu \in N(T^{q+k}) & = N(T^{q})\\
							T^{q}(Tu) & = T^{q+1} u= 0 \\
							u \in N(T^{q+1}) &= N(T^{q})
						\end{align*}
				\end{description}
		\end{itemize}
		Ahora tomamos $U = N(T^{q})\wedge W = Im(T^{q}) = T^{q}V$
		\begin{itemize}
			\item $U$ invariante por $T$
				$$T(U)\subseteq N(T^{q+1}) = N(T^{q}) = U$$
			\item $W$ invariante por $T$.
				Sea $u\in W = T^{q}V$ 
				\begin{align*}
					& \Rightarrow\exists v\in V/ T^{q}v = u\\
					& \Rightarrow Tu = T(T^{q}v) = T^{q}(Tu)\in T^{q}V = W
				\end{align*}
				$$\therefore T(W)\subseteq W$$				
		\end{itemize}
		Afirmamos que la suma $U+W$ es suma directa, pues si $v\in U\cap W$, entonces:
		\begin{align*}
			\Rightarrow v\in N(T^{q})\quad &\wedge\quad v\in T^{q}V = W\\
			\Rightarrow	T^{q}v = 0\quad &\wedge\quad\exists w\in V/ v = T^{q}w\\
			T^{2q}w & = T^{q}v = 0 \\
			\Rightarrow w\in N(T^{2q}) & = N(T^{q})\\
			\Rightarrow T^{q}w & = 0 = v 
		\end{align*}
		Por el teorema de la dimensión en $T^{q}:V\rightarrow V$, tenemos:
		$$\dim V = \dim T^{q}V  + \dim N(T^{q}) = \dim (T^{q}V\oplus N(T^{q})) = \dim (U\oplus W)$$
		
		Por lo que $V = T^{q}V \oplus N(T^{q}) = U\oplus W$.
		
		Finalmente probemos que $T_{|U}:U\rightarrow U$ es nilpotente y $T_{|W}:W\rightarrow W$ es inversible.
		\begin{itemize}
			\item $T_{|U}:U\rightarrow U$ nilpotente.
			Como $q\in\mathbb{N}$, es el mínimo tal que $ N(T^{q}) = N(T^{q+1})$
			
			Entonces $N(T^{q-1}) \neq \emptyset$, pues si $N(T^{q-1}) = \emptyset$ implicaría que $N(T^{q}) = \emptyset$. Para probar esto supongamos que $N(T^{q})\neq \emptyset$.
			$$\exists v\in N(T^{q}),\: v\neq 0/\quad T^{q}v = 0$$
			\begin{align*}
				T^{q-1}(Tv) &= 0\\
				Tv \in N(T^{q-1}) &= \{0\}\\
				Tv & = 0\\
				v & \in N(T)\subseteq N(T^{q-1})\\
				v &\in N(T^{q-1})
			\end{align*}
			Contradice que $N(T^{q-1}) = \emptyset$, por lo que $N(T^{q}) = \emptyset$
			\begin{equation}\label{minimalidad_index}
				\Rightarrow N(T^{q-1}) = N(T^{q})
			\end{equation}
			Pero $(\ref{minimalidad_index})$, contradice la minimalidad de $q\in\mathbb{N}$
			$$\therefore N(T^{q-1})\neq \emptyset$$
			Así tenemos $T^{q} v = 0,\quad\forall v\in U\quad\wedge\quad\exists u\in U/\quad T^{q-1}u\neq 0$
			$$\Rightarrow T_{|U}^{q-1}\neq 0\quad\wedge\quad T_{U}^{q} = 0$$
			$\therefore T_{|U}$ es nilpotente de índice $q$.
			
			\item $T_{|W}:W\rightarrow W$ inversible.
			Sea $v\in N(T_{|W})\subseteq W = T^{q}V$
			\begin{align*}
				\Rightarrow Tv &= 0\\
				pero\; v &\in T^{q}(V)\\
				\Rightarrow \exists u &\in V/\quad T^{q}u = v\\
				T^{q+1}u &= 0\\
				\Rightarrow u &\in N(T^{q+1}) = N(T^{q})\\
				\Rightarrow T^{q}u &= 0 = v
			\end{align*}
			Por lo que $N(T_{|W}) = \emptyset$, asi $T_{|W} $ es inversible.
		\end{itemize}
\end{itemize}


\begin{mytheo}[Unicidad de la descomposición de $T$]
La descompocisión de $T$ en una nilpotente y en una inversible es única.	
\end{mytheo}

\textbf{Prueba(pendiente)}\\


\begin{mytheo}[Forma Canónica de Jordan]
	Sea $V$ un $\mathbb{C}$-espacio vectorial, $\dim V = n$, $T:V\rightarrow V$ un operador lineal y $\lambda_{1},\lambda_{2},\ldots,\lambda_{k}\in \wedge (T)$ y $n_{1}, n_{2},\ldots , n_{k}$ las multiplicidades algebráicas de los valores propios(respectivamenete), entonces:

Existen subespacios $V_{1}, V_{2},\ldots, V_{k}$ invariantes por $T\;\left(T(V_{i})\subseteq V_{i}\right)/$
\begin{enumerate}
	\item	$V = V_{1}\oplus V_{2}\oplus\ldots\oplus V_{k}$
	\item $\dim V_{i} n_{i},\quad i=1-k$
	\item El operador $T-\lambda_{i}I:V_{i}\rightarrow V_{i}$ es nilpotenete, $i=1-k$.
\end{enumerate}
\end{mytheo}




\end{document}