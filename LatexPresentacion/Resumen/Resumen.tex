\documentclass[10pt,a4paper]{article}
%\usepackage[english,spanish]{babel}
\usepackage{indentfirst}
\usepackage{anysize} % Soporte para el comando \marginsize
%\marginsize{1.5cm}{1.5cm}{0.5cm}{1cm}
\marginsize{2,5cm}{1,8cm}{4cm}{1,7cm}
\usepackage[psamsfonts]{amssymb}
\usepackage{amssymb}
\usepackage{amsfonts}
\usepackage{amsmath}
\usepackage{amsthm}
\usepackage{stackrel}
\usepackage{graphicx}
\usepackage[colorinlistoftodos]{todonotes}
%Color a las referencias
\usepackage[colorlinks=true, allcolors=blue]{hyperref}
\usepackage[spanish]{babel}
\selectlanguage{spanish}
\usepackage[utf8]{inputenc} 

\usepackage{multicol}
\renewcommand{\thepage}{}
\columnsep=7mm

%%%%%%%%%%%%%%%%%%%%%%%%%%%%%%%%%%%%%%%%
\newtheorem{definicion}{Definici\'on}[section]
\newtheorem{teorema}{Teorema}[section]
\newtheorem{prueba}{Prueba}[section]
\newtheorem{prueba*}{Prueba}[section]
\newtheorem{corolario}{Corolario}[section]
\newtheorem{observacion}{Observaci\'on}[section]
\newtheorem{lema}{Lema}[section]
\newtheorem{ejemplo}{Ejemplo}[section]
\newtheorem{solucion*}{Soluci\'on}[section]
\newtheorem{algoritmo}{Algoritmo}[section]
\newtheorem{proposicion}{Proposici\'on}[section]

\linespread{1.4} \sloppy

\newcommand{\R}{\mathbf{R}}
\newcommand{\N}{\mathbf{N}}
\newcommand{\C}{\mathbb{C}}
\newcommand{\Lr}{\mathcal{L}}
\newcommand{\fc}{\displaystyle\frac}
\newcommand{\ds}{\displaystyle}

\DeclareMathOperator{\Dom}{Dom}

%%%%%%%%%%%%%%%%%%%%%%%%%%%%%%%%%%%%%%%%

\renewcommand{\thefootnote}{\fnsymbol{footnote}}
\usepackage{url}
\usepackage{hyperref}

\begin{document}
\begin{center}
 {\Large \textbf{TIRO EN DIANA}}
\end{center}
\begin{center}
 Gustavo Lozano$^{1}$, Miller Silva$^{2}$, Ayrton Coronado$^{3}$, Mirian Geronimo$^{4}$, Guillermo Borjas$^{5}$ \vskip5pt
 {\it Facultad de Ciencias$^1$, Universidad Nacional de Ingenier\'{\i}a$^1$\\}\vskip5pt
 Email: glozanoa@uni.pe$^{1}$, miller.silva.m@uni.pe$^{2}$, acoronadoh@uni.pe$^{3}$, mgeronimoa@uni.pe$^{4}$, gborjasc@uni.pe$^{5}$
\end{center}
%\maketitle 
\vspace*{1cm}
\begin{abstract}

\noindent El estudio de las matrices es fundamental para todo aquel que desea sumergirse en el maravilloso mundo de las matemáticas, ya que las matrices se encuentran en áreas como el álgebra lineal, el análisis numérico y las probabilidades. Las matrices tienen varios usos en matemática que van desde el ordernamiento datos hasta incluso la definición de nuevas  funciones como $\sin(A),e^A, a_0+a_1A+\ldots+a_nA^n$; otro uso muy frecuente es la potenciación de matrices, usual en grafos (matriz de adyacencia). Imaginemos que quisiéramos hallar la potencia diez de una matriz $A\in\mathbb{R}^{12\times 12} $, querer calcularlo directamente sería una cosa de locos; para estos casos los matemáticos recomiendan trabajar con una matriz diagonal que sea semejante a $A$ esto es $A=PDP^{-1}$ la cual verifica $A^n=PD^nP^{-1}$, donde hallar la $D^n$ resulta mucho más fácil que $A^n$. Con esto podríamos estar tranquilos pero ¿Qué pasa si la matriz no es semejante a ninguna matriz diagonal?, aún podemos guardar la calma ya que en este caso podemos usar la forma canónica de Jordan de la matriz $A$, con el que también se cumple que $A^n=PJ^nP^{-1}$. \textit{La matriz de Jordan} es una bonita herramienta matemática que nos ayuda a simplificar operaciones, para nuestra suerte ya se conoce la forma general de potencia $n-$ésima de la matriz de Jordan, lo ``difícil'' ahora es saber cómo hallarlo; para hallarlo es necesario calcular los valores propios de $A$, esto lo podemos hacer usando herramientas del análisis numérico como \textit{el método potencia, potencia inversa y QR}. Este trabajo se centra en calcular la potencia $k-$ésima de una matriz $A$ (\textit{la matriz de puntuación}), para esto usaremos lo mencionado lineas arriba.
\end{abstract}

\begin{quotation}
	{\small
		\noindent\textbf{Palabras Clave:} \\ 
	La matriz de Jordan, Método potencia , Método potencia inversa, Método QR, Matriz de puntuación \\
	}
\end{quotation}

\renewcommand{\abstractname}{Abstract}
\begin{abstract}
	\noindent The study of matrices is fundamental for anyone who wants to immerse themselves in the wonderful world of mathematics, since matrices are found in areas such as linear algebra and numerical analysis and the probabilities. Matrices have several uses in mathematics ranging from data processing to the definition of new functions such as $\sin(A),e^A, a_0+a_1A+\ldots+a_nA^n$; another very frequent use is matrix enhancement, usual in graphs (adjacency matrix). Imagine that we would like to find the ten power of a matrix $A\in\mathbb{R}^{12\times 12} $, to want to calculate it directly would be a crazy thing; for these cases, mathematicians recommend working with a diagonal matrix that is similar to $A$ this is $A=PDP^{-1}$ which verifies $A^n=PD^nP^{-1}$ , where to find the $D^n$ it's much easier than $A^n$. With this we could be calm but ¿What happens if the matrix is not similar to any diagonal matrix? We can still keep calm because in this case we can use Jordan's canonical form of the matrix  $A$, with which it is also true that $A^n=PJ^nP^{-1}$. \textit{The matrix of Jordan} is a nice mathematical tool that helps us to simplify operations, for our luck we already know the general form of $n$-th power of the matrix of Jordan, the `` difficult '' now is to know how to find it; to find it, it is necessary to calculate the eigenvalues of $A$, this can be done using numerical analysis tools such as \textit{the power method, inverse power and QR}. This work focuses on calculating the k-th power of a matrix A (\textit{the scoring matrix}), for this we will use the aforementioned lines.
\end{abstract}


\begin{quotation}
	{\small
		\noindent \textbf{Keywords:} \\ 
		The matrix of Jordan, The power method , The inverse power method, The method QR, the scoring matrix \\
	}
\end{quotation}



\begin{abstract}
	
	\noindent El estudio de las matrices es fundamental para todo aquel que desea sumergirse en el maravilloso mundo de las matemáticas. Las matrices son parte escencial de ésta y son ampliamente usadas en diversas áreas como lo es en el álgebra lineal, el análisis numérico y las probabilidades. Por ejemplo, para modelar el comportamiento de la propagación de un virus ``T'', se usa la potenciación de matrices; supongamos que tenemos una población de personas sanas ($S$) e infectadas ($I$) donde las personas sanas tienen un $80\%$ de probabilidad de no infectarse en 2 días, y que además hay una cura con $1\%$ de efectividad, estos datos se ordenan en una matriz $A$. Luego si empezamos con una población de 99 personas sanas y 1 infectado, dentro de 2 días tedríamos que la nueva población es: 
	\begin{equation*}
	\begin{pmatrix} S\\I \end{pmatrix}_2 
	=
	\begin{pmatrix} 0.8 & 0.01\\0.2&0.99 \end{pmatrix} 
	\begin{pmatrix} 99\\ 1 \end{pmatrix}_0
	=
	\begin{pmatrix} 79.21 \\ 20.79 \end{pmatrix}_2, 
	\end{equation*}
	es decir que aproximadamente tendríamos 79 personas sanas y 21 infectadas; y ¿qué pasaría en $2n$ días?, bueno tendríamos que la población sería $A^nX_0$. Es aquí donde comienzan las complicaciones, dado que elevar una matriz a su n-ésima potencia no es tarea fácil, sin embargo existe una maravillosa herramienta matemática que nos ayudará a simplificar los cálculos; nos referimos a la matriz de Jordan. La matriz de Jordan satisface $A^n=PJ^nP^{-1}$, lo cual aplicando esto a nuestro problema anterior tendríamos que:
	\begin{equation*}
	A^n=
	\begin{pmatrix} 0.8 & 0.01\\0.2&0.99 \end{pmatrix}^n
	=
	\begin{pmatrix} -0.05&-0.71\\ -1.00 &0.71 \end{pmatrix}
	\begin{pmatrix} 1&0\\ 0 &0.79  \end{pmatrix}^n
	\begin{pmatrix} -0.05&-0.71\\ -1.00 &0.71 \end{pmatrix}^{-1}
	\end{equation*}
	ahora si $n$ es muy grande obtenemos:
	\begin{equation*}
	\begin{pmatrix} 1&0\\ 0 &0.79  \end{pmatrix}^n
	\rightarrow
	\begin{pmatrix} 1&0\\ 0 &0 \end{pmatrix}
	\Rightarrow
	A^n\rightarrow \begin{pmatrix} 0.05&0.05\\ 0.95 &0.95 \end{pmatrix}
	\end{equation*}
	finalmente cuando $n\to\infty$ la población esperada es:
	\begin{equation*}
	\begin{pmatrix} S\\I \end{pmatrix}_\infty
	=
	\begin{pmatrix} 0.05&0.05\\ 0.95 &0.95 \end{pmatrix}
	\begin{pmatrix} 99\\1 \end{pmatrix}
	=\begin{pmatrix} 4.76\\95.24 \end{pmatrix},
	\end{equation*}
	esto es, 5 personas sanas y 95 infectadas; ¿increíble verdad?.
	
	Vemos así que \textit{la matriz de Jordan} es muy útil para simplificar operaciones; el desafío es hallarla; para esto es necesario calcular los valores propios de $A$, los cuales podemos obtener usando algunos métodos del análisis numérico tales como \textit{el método potencia, potencia inversa y QR}. Para nuestra suerte ya se conoce la forma general de la potencia  $n-$ésima de la matriz de Jordan.
	
	Este trabajo se centra en calcular la potencia $k-$ésima de una matriz $A\in\R^{12\times 12}$ (\textit{la matriz de puntuación}), la cual contiene puntuaciones de una prueba de tiro en dianas realizados por un cadete, para esto calcularemos su matriz de Jordan asociada, ayudándonos de los métodos numéricos ya mencionados y algunos otros algoritmos de apoyo para calcular los valores propios y los bloques de Jordan.
\end{abstract}

\begin{quotation}
	{\small
		\noindent\textbf{Palabras Clave:} \\ 
		La matriz de Jordan, Método potencia , Método potencia inversa, Método QR, Matriz de puntuación \\
	}
\end{quotation}

\renewcommand{\abstractname}{Abstract}
\begin{abstract}
	\noindent The study of matrices is fundamental for those who wish to immerse themselves in the wonderful world of mathematics, since matrices are in areas such as Linear Algebra, Numerical Analysis and Probabilities. For example to model the behavior of virus propagation T, it use matrices potentiation, suppose we have a population healthy($S$) and infected$(I)$, with an $80\%$ chance that healthy people will not become infected in two days, and also there is a cure with $1\%$ effectiveness, hese data are ordered in a matrix $A$. Then if we have a population of 99 healthy persons and 1 infected person, 2 days later we would have that the new population is:
	$$
	\begin{pmatrix} S\\I \end{pmatrix}_2 
	=
	\begin{pmatrix} 0.8 & 0.01\\0.2&0.99 \end{pmatrix} 
	\begin{pmatrix} 99\\ 1 \end{pmatrix}_0
	=
	\begin{pmatrix} 79.21 \\ 20.79 \end{pmatrix}_2, 
	$$
	this is approximately we would have 79 healthy persons and 21 infected persons, and ¿What would happen with the population in 2 days?, well the population would be $A^{n}X_{0}$. It's here where
	things get complicated, but we can use a wonderful mathematical tool  that would help us simplify the calculations, we are talking about the matrix of Jordan, which satisfies $A^{n}=PJ^{n}P^{-1}$, so we would have:\\
	\begin{equation*}
	A^n=
	\begin{pmatrix} 0.8 & 0.01\\0.2&0.99 \end{pmatrix}^n
	=
	\begin{pmatrix} -0.05&-0.71\\ -1.00 &0.71 \end{pmatrix}
	\begin{pmatrix} 1&0\\ 0 &0.79  \end{pmatrix}^n
	\begin{pmatrix} -0.05&-0.71\\ -1.00 &0.71 \end{pmatrix}^{-1}
	\end{equation*}
	Now, if $n$ is a very large number, we would have:\\
	$$
	\begin{pmatrix} 1&0\\ 0 &0.79  \end{pmatrix}^n
	\rightarrow
	\begin{pmatrix} 1&0\\ 0 &0 \end{pmatrix}
	\Rightarrow
	A^n\rightarrow \begin{pmatrix} 0.05&0.05\\ 0.95 &0.95 \end{pmatrix}
	$$
	Finally, when $n\to\infty$ the expected population is:\\
	$$
	\begin{pmatrix} S\\I \end{pmatrix}_\infty
	=
	\begin{pmatrix} 0.05&0.05\\ 0.95 &0.95 \end{pmatrix}
	\begin{pmatrix} 99\\1 \end{pmatrix}
	=\begin{pmatrix} 4.76\\95.24 \end{pmatrix},
	$$
	this is 5 healthy persons and 95 persons infected. That's incredible, right?!
	
	\noindent So, we see that \textit{Jordan matrix} is very useful to simplify operations; the ``complication''
	is to find it; for this it's necessary to calculate the eigenvalues of $A$, this we can do
	using numerical analysis tools such as \textit{the power method, inverse power and
		QR}. For our luck we already know the general form of $J^{n}$, with $J$:
	Jordan matrix.
	
	\noindent This work focuses on calculating the power $k$ of a matrix $A \in \mathbb{R}^{12\times12}$
	(\textit{the scoring matrix}), where $A$ contains  the scores of the shooting test of a
	cadet, for this we will use the Jordan Form and the numerical methods, indicated
	lines above, to calculate the eigenvalues and blocks' Jordan.
\end{abstract}

\begin{quotation}
	{\small
		\noindent \textbf{Keywords:} \\ 
		The matrix of Jordan, The power method , The inverse power method, The method QR, The scoring matrix \\
	}
\end{quotation}
\end{document}